% A simple XeLaTeX template for typesetting Church Slavonic texts with their
% English translations paragraph by paragraph, using the Church Slavonic,
% Polyglossia, and Paracol packages.
% Made for my friend Ben.

\documentclass[a4paper]{article}
\thispagestyle{empty} % suppress page numbers

\usepackage{polyglossia}
   \setmainlanguage{english}
   \setotherlanguage{churchslavonic}

\setromanfont{Garamond}
\newfontfamily\churchslavonicfont[Script=Cyrillic, Ligatures=TeX, HyphenChar=_]
{PonomarUnicode.otf} 
% Church Slavonic fonts provided in churchslavonic package:
% {Fedorovsk, Indiction, Menaion, Monomakh, Ponomar}Unicode.otf

\usepackage{churchslavonic}

\setlength{\parskip}{0.5\baselineskip} % paragraph vertical space
\setlength{\parindent}{0pt} % no paragraph indents

% no hyphenation:
\tolerance=1
\emergencystretch=\maxdimen
\hyphenpenalty=10000
\hbadness=10000

\raggedright % rather than justified

\usepackage[margin=0.8in]{geometry} % page margins
\usepackage{paracol} % parallel columns
\usepackage{xcolor} % text and page colors

\newcounter{languageSwitch}
\newcommand\switch{
   \stepcounter{languageSwitch}
   \ifodd\value{languageSwitch}
      \switchcolumn
      \selectlanguage{english}
   \else
      \switchcolumn*
      \selectlanguage{churchslavonic}
   \fi
   \stepcounter{languageSwitch}
}

\begin{document}
\pagecolor[HTML]{FFFEE2} % cream background
\setlength{\columnsep}{20pt} % separation between columns
\begin{paracol}{2} % begin two parallel columns
\selectlanguage{churchslavonic}
мѣсѧца дєкѧбр҄ја иг въ навєчєриѥ рождьства хрьстова єванћєлиѥ отъ лѹкъі глава 
в въ оно врѣмѧ изідє заповѣдь отъ кєсарѣ авгоста напісаті в҄сѫ вьсєлєнѫѭ

\switch
And it came to pass in those days, that there went out a decree from Caesar 
Augustus, that all the world should be taxed.

\switch
сє напісаніє пръвоє бъістъ владѫщѹ сѹрієѭ и кѵрінієѭ

\switch
(And this taxing was first made when Cyrenius was governor of Syria.)

\switch
и идѣахѫ вьсі напісатъ сѧ кьждо въ свои градъ

\switch
And all went up to be taxed, every one in his own city.

\switch
вьзідє жє иосіфь отъ галілєѧ и града назарєтьска вь июдєѭ вь градъ давъідовъ 
іжє наріцаєтъ сѧ віѳлєємь занє бѣашє отъ домѹ и отьчьствіѣ давъідова

\switch
And Joseph also went up from Galilee, out of the city of Nazareth, 
into Judaea, unto the city of David, which is called Bethlehem; 
(because he was of the house and lineage of David:)

\switch
напісатъ сѧ съ марієѭ обрѫчєнѫѭ ємѹ жєноѭ сѫштєѭ нєпраздъноѭ

\switch
To be taxed with Mary his espoused wife, being great with child.

\switch
бъістъ жє єгда бъістє тѹ исплънишѧ сѧ дєниє да родітъ

\switch
And so it was, that, while they were there, the days were accomplished 
that she should be delivered.

\switch
и роді съінъ свои пръвѣнєць и обитъі и и положі и въ ѣслєхъ занє нє бѣ 
има мѣста въ обитѣли

\switch
And she brought forth her firstborn son, and wrapped him in swaddling clothes, 
and laid him in a manger; because there was no room for them in the inn.

\switch
и пастъирі бѣахѫ въ тоиждє ст҄ранѣ бъдѧщє и 
стрѣгѫщє стражѫ нощьнѫѭ о стадѣ своємъ

\switch
And there were in the same country shepherds abiding in the field, 
keeping watch over their flock by night.

\end{paracol}
\end{document}